%-------------------------------------------------------------------------------
% yum_resource_usage
%-------------------------------------------------------------------------------
%
% \file        yum_resource_usage.tex
% \library     Documents
% \author      Chris Ahlstrom
% \date        2017-09-24
% \update      2018-03-17
% \version     $Revision$
% \license     $XPC_GPL_LICENSE$
%
%     Provides a very incomplete description and discusson of Yoshimi
%     resource usage.
%
%-------------------------------------------------------------------------------

\section{Resource Usage}
\label{sec:resource_usage}

   This section discusses the resource usage (and hence, efficiency) of
   \textsl{Yoshimi}.

   In the meantime, we include some notes.

\subsection{Resource Usage / JACK Buffer}
\label{sec:resource_usage_jack_buffer}

   If \textsl{Yoshimi}'s internal buffer is \textsl{smaller} than the JACK
   buffer, it sounds quite horrible. If it's the same or greater there's no
   problem. The reason we didn't find this before is that it only affects
   \textsl{Yoshimi} resource_usage.  Also it doesn't apply to any of the
   released versions.  The cause of the problem is that the resource_usage code
   doesn't have the same looping structure that was added to the standalone
   routine to deal with exactly this situation (re-entering the audio
   'construction' function until the JACK buffer is filled). We hope Andrew can
   deal with this fairly soon as we don't understand the resource_usage code
   very well.  There are valid reasons for wanting different sizes for these
   buffers. The internal buffer size as well as affecting latency and CPU load
   also alters the sound in quite subtle ways. It particularly affects the
   behaviour of filters.  One day we may be able to stop this happening, but in
   the mean time we have to live with it.  For our purposes, we find a buffer
   size of 128 or 256 is best.

   Anecdote from Will:

   To get some idea of how memory was being allocated I loaded up 16 parts all
   with 'Ghost Ensemble', then using Rosegarden created a file using all 16
   parts at 5 notes each, starting the notes one at a time, then keeping all of
   them on for about half a minute. This is the maximum number of notes at any
   one time that Yoshimi supports.

   The first thing I found was that I had to go up to 256 frames per period to
   run this (I normally run at 32 these days). That wasn't really a surprise,
   what \textsl{was} a surprise was a rather crude estimation of the memory
   usage.

   Going from an empty \textsl{Yoshimi} to this 16 part monster stole an
   astonishing 100M. Now to be fair, this patch adds all three engines, so
   that's 16 lots of PADsynth samples along with 16 ADDsynth oscillators.

   Actually playing these 80 notes only took another 2M. Now this difference
   was quite consistent, even over reboots, although when the notes stop only
   1M is given back. This isn't a memory leak as playing again only re-grabs
   another 1M, so I'm guessing that the underlying OS memory management isn't
   too fussy about garbage collection until it really needs it.

   Now if we multiply that by 4 that's 8M for the default max for each part,
   which is quite a lot (and much more than the current overall limit), but
   probably quite do-able on all modern machines.

\subsection{Resource Usage / JACK and ALSA Buffer Sizes}
\label{sec:resource_usage_buffer_sizes}

   This section elaborates on
   \sectionref{paragraph:menu_yoshimi_settings_main_settings}.

   If something in the JACK connection
   graph is slow enough to need a large buffer
   size, but is not directly related to \textsl{Yoshimi},
   then having a smaller internal
   buffer size will enable \textsl{Yoshimi}
   to interlace incoming MIDI messages more accurately.

   If working ALSA MIDI with JACK Audio, a smaller buffer size will allow
   incoming MIDI to be better placed; MIDI is then buffered and it will read
   the buffer every time it loops round. This is at the cost of CPU
   availability.

   If working JACK MIDI with ALSA audio? There's no benefit having the
   two buffer sizes different.

   If entirely ALSA, \textsl{Yoshimi} \textsl{defines}
   the buffer size, and hence the latency.
   However, there is a quirk... if one accidentally sets all ALSA
   (i.e. forgets to start JACK),
   then ALSA sets the buffer size, usually a massive 1024 frames.

   In an LV2 plug-in, internal buffer size is ignored.

%-------------------------------------------------------------------------------
% vim: ts=3 sw=3 et ft=tex
%-------------------------------------------------------------------------------
