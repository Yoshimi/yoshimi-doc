%-------------------------------------------------------------------------------
% yum_manpage
%-------------------------------------------------------------------------------
%
% \file        yum_manpage.tex
% \library     Documents
% \author      Chris Ahlstrom
% \date        2015-06-22
% \update      2017-03-27
% \version     $Revision$
% \license     $XPC_GPL_LICENSE$
%
%     Provides the man page section of yoshimi-user-manual.tex.
%
%-------------------------------------------------------------------------------

\section{Yoshimi Man Page}
\label{sec:yoshimi_man_page}

   This version \textsl{Yoshimi} manula page is actually the output of the
   \texttt{yoshimi --help} command, which prints out the command-line that
   are discussed in this section.  Note that further descriptions might be
   found in other sections of this advanced user manual, for example, in
   \sectionref{subsubsec:menu_yoshimi_settings}.

   Yoshimi 1.5.1 (and above!), a derivative of ZynAddSubFX - Copyright
   2002-2009 Nasca Octavian Paul and others, Copyright 2009-2011 Alan Calvert,
   Copyright 20012-2013 Jeremy Jongepier and others,
   Copyright 20014-2017 Will Godfrey and others.

   \setcounter{ItemCounter}{0}      % Reset the ItemCounter for this list.

  \optionpar{-a}{--alsa-midi[="device"]}
      Use ALSA MIDI input.
      From the command line, as well as autoconnecting the main L \& R
      outputs to JACK, with ALSA MIDI one can now auto-connect to a known source.

   \begin{verbatim}
      ./yoshimi -K --alsa-midi="Virtual Keyboard"
   \end{verbatim}

      ALSA can often manage with just the client name.  This command is case
      sensitive, and quite fussy about spaces, etc., so it's wise to use
      quotes for the source name, even if they don't seem to be needed.

  \optionpar{-A}{--alsa-audio[=device]}
      Use ALSA audio output.

  \optionpar{-b}{--buffersize=size}
      Set ALSA internal audio buffer size.

  \optionpar{-c}{--show-console}
      Show the console on startup.

  \optionpar{-D}{--define-root}
      Define the path to a new bank root.
      \textsl{Yoshimi} will then immediately scan this path for new banks,
      but won't make the root (or any of its banks) current. The final
      directory doesn't in fact have to be 'banks' but traditionally
      \textsl{Yoshimi} has always done this.
      Also, when running from the command line there is now access to many of
      the system, root, bank, etc. settings.
      See \sectionref{sec:command_line}.

  \optionpar{-i}{--no-gui}
      Do not show the GUI.  See \sectionref{sec:command_line} for more
      information about this mode of operation.  Note that the command-line
      and the GUI can be available simultaneously.  Also note that this
      switch allows \textsl{Yoshimi} to run on a dumb terminal or virtual
      console.

  \optionpar{-j}{--jack-midi[=device]} 
      Use JACK MIDI input.
      From the command line, as well as autoconnecting the main L \& R
      outputs to JACK, with JACK MIDI one can now auto-connect to a known source.

   \begin{verbatim}
      ./yoshimi -K --jack-midi="jack-keyboard:midi_out"
   \end{verbatim}
   
   JACK needs the port as well as the name.
   This command is case sensitive, and fussy about spaces.
   Use quotes.

  \optionpar{-J}{--jack-audio[=server]}
      Use JACK audio output.
      Connect to the given JACK server if given.

  \optionpar{-k}{--autostart-jack}
      Auto-start the JACK server.
      Note that this can cause some odd behavior on some systems, so be aware of
      that possibility.

  \optionpar{-K}{--auto-connect}
      Auto-connect JACK audio.  Note that, if the auto-connect feature has been
      specified in the user-interface, and saved to the \textsl{Yoshimi}
      configuration file, there is then no way to disable this feature from the
      command-line, at this time.

  \optionpar{-l}{--load=file}
      Load an \texttt{.xmz} file.

  \optionpar{-L}{--load-instrument=file}
      Load an \texttt{.xiz} file  The '=' is optional, but must not be
      surrounded by spaces if present.

  \optionpar{-N}{--name-tag=tag}
      Add tag to client-name.

  \optionpar{-o}{--oscilsize=size}
      Set ADDSynth oscillator size (OscilSize).

  \optionpar{-R}{--samplerate=rate}
      Set ALSA audio sample rate.

  \optionpar{-S}{--state[=file]}
      Load saved state from file, where the file defaults to
      \texttt{\$HOME/.config/yoshimi/yoshimi.state}, which is loaded
      automatically if present and not overridden by the \texttt{--state}
      option. The '=' is mandatory, no spaces allowed.

  \optionpar{-u}{--jack-session-file[=file]}
      Load the named JACK session file.

  \optionpar{-U}{--jack-session-uuid[=uuid]}
      Load the named JACK session by UUID.

  \optionpar{-?}{--help}
      Give this help list.

   \optionpar{--usage}
      Provide a short usage message.

  \optionpar{-V}{--version}
      Print the program version.

   Mandatory or optional arguments to long options are also mandatory or
   optional for any corresponding short options.

   From the command line, as well as autoconnecting the main L \& R outputs
   to JACK, with either JACK or ALSA MIDI one can now auto-connect to a
   known source.

   ALSA can often manage with just the client name, but JACK needs the port
   as well. These commands are case sensitive, and fussy about spaces.

%-------------------------------------------------------------------------------
% vim: ts=3 sw=3 et ft=tex
%-------------------------------------------------------------------------------
