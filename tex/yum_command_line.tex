%-------------------------------------------------------------------------------
% yum_command_line
%-------------------------------------------------------------------------------
%
% \file        yum_command_line.tex
% \library     Documents
% \author      Chris Ahlstrom
% \date        2015-09-06
% \update      2015-09-06
% \version     $Revision$
% \license     $XPC_GPL_LICENSE$
%
%     Provides the description of the no-gui mode of Yoshimi.
%
%-------------------------------------------------------------------------------

\section{The Yoshimi Command Line Interface}
\label{sec:command_line}

   \textsl{Yoshimi} provides a "no GUI" or "command-line" mode of operation
   where some aspects of the application can be controlled via textual commands.
   This mode is useful for blind people, for example.

   One of the main features of the 1.3.6 release is improved non-GUI
   accessibility.  This feature focusses on using two universally available
   techniques. The effectiveness (and indeed usefulness) of this feature
   will help shape future complementary interfaces. Also, a number of
   first-time defaults have been changed to make this feataure easier.

   When starting from the command line, an argument can be included for a new
   root path to be defined to point to a set of banks fetched from elsewhere.
   This will be given the next free ID. A future upgrade will allow the ID to
   be set to any valid one when it is created, mirroring the GUI behaviour.

\subsection{Commands}
\label{subsec:command_line_commands}

   When running from the command line, the commands
   (see \tableref{table:yoshimi_text_commands})
   can be entered after the 'up and running' message.

   \begin{table}[H]
      \centering
      \caption{Yoshimi Text Commands}
      \label{table:yoshimi_text_commands}
      \begin{tabular}{l l}
         \texttt{setup} &
            Displays the current dynamic system settings. \\

         \texttt{show paths} &
            Displays all the currently defined bank root paths and their IDs. \\

         \texttt{list root (n)} &
            Lists all the banks and their IDs in root path 'n' (or the
            current root). \\

         \texttt{list bank (n)} &
            Lists all the instruments and their IDs in bank 'n' (or the
            current bank) of the current root. \\

         \texttt{set rootcc (n)} &
            Set the MIDI CC for root path changes (128 disables). \\

         \texttt{set bankcc (n)} &
            Set the MIDI CC for bank changes (anything other than 0 or 32
            disables it). \\

         \texttt{set path (n)} &
            Set current root path to ID 'n'. \\

         \texttt{set bank (n)} &
            Set current bank to ID 'n'. \\

         \texttt{set part (n1) program (n2)} &
            Load instrument 'n2' into part 'n1'. \\

      \end{tabular}
   \end{table}

   Commands are not case sensitive and an invalid one will print a reminder.
   usually, one only needs the first 4 letters of the names provided that is
   unabiguous. i.e. 'rootcc' has to be in full so it isn't confused with
   'root'.

   More commands will be added, and the organisation of the commands
   may change slightly.

%-------------------------------------------------------------------------------
% vim: ts=3 sw=3 et ft=tex
%-------------------------------------------------------------------------------
