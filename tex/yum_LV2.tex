%-------------------------------------------------------------------------------
% yum_LV2
%-------------------------------------------------------------------------------
%
% \file        yum_LV2.tex
% \library     Documents
% \author      Chris Ahlstrom
% \date        2015-10-25
% \update      2016-05-24
% \version     $Revision$
% \license     $XPC_GPL_LICENSE$
%
%     Provides a very incomplete description and discusson of Yoshimi LV2
%     support.
%
%-------------------------------------------------------------------------------

\section{LV2 Plug-in Support}
\label{sec:lv2_plugin}

   \textsl{Yoshimi} now runs as an LV2 plugin.

Supported features:

   \begin{enumber}
      \item Sample-accurate midi timing.
      \item State save/restore support via LV2\_State\_Interface.
      \item Working UI support via LV2\_External\_UI\_Widget.
      \item Programs interface support via LV2\_Programs\_Interface.
      \item Multi channel audio output. 'outl' and 'outr' have LV2 index 2
         and 3. All individual ports numbers start at 4.
   \end{enumber}

   Planned feature: Controls automation support. This will be a part of a
   common controls interface.

   Download and build the source code found at the
   \textsl{Yoshimi} site \cite{yoshimi},
   and one will find a file named
   \texttt{LV2\_Plugin/yoshimi\_lv2.so}

% Just list discussion, and never intended to be a permanent record. Also it is
% now completely untrue! It has been untrue since Version 1.3.6 was released at
% the end of September 2015.
%
%  The LV2 \textsl{Yoshimi} interface can be run in hosts such as
%  \textsl{Ardour 3}, \textsl{Carla}, and \textsl{QTractor}.  We have tested
%  LV2 using the latest versions of \textsl{Ardour}, \textsl{Muse} and
%  \textsl{Qtractor} as LV2 hosts. We also tried \textsl{Carla} but couldn't
%  get anywhere with it at all... if someone is familiar with it we'd be
%  interested in what results they get.  Slightly to our surprise the one that
%  performed the best in every respect was \textsl{Qtractor}. \textsl{Muse} got
%  everything correct, but was prone to Xruns on program changes.
%  \textsl{Ardour} was very problematical.  There were no Xruns but it seemed
%  to have odd timing issues. Also, on two occasions it managed to shorten the
%  decay times of two of the instruments.  We don't understand how it managed
%  that.  Our reference was the original MIDI file played into a stand-alone
%  \textsl{Yoshimi} via \texttt{aplaymidi}. This also behaves identically to
%  the file being sequenced by \textsl{Rosegarden}.

   \textsl{Yoshimi}'s LV2 impementation is frequently tested using the latest
   versions of \textsl{Ardour}, \textsl{Muse}, and \textsl{Qtractor} as LV2
   hosts. Like \textsl{Yoshimi}, these are also in continuous development. So
   far we've not been able to get anywhere with \textsl{Carla}; if someone is
   familiar with it, we'd be interested in what results they get.

   At some point we hope to document the process of setting up and using
   the \textsl{Yoshimi} LV2 plugin.

%  In the meantime, we include some notes.

%  If \textsl{Yoshimi}'s internal buffer is \textsl{smaller} than the JACK
%  buffer, it sounds quite horrible. If it's the same or greater there's no
%  problem. The reason we didn't find this before is that it only affects
%  \textsl{Yoshimi} LV2.  Also it doesn't apply to any of the released
%  versions.  The cause of the problem is that the LV2 code doesn't have the
%  same looping structure that was added to the standalone routine to deal with
%  exactly this situation (re-entering the audio 'construction' function until
%  the JACK buffer is filled). We hope Andrew can deal with this fairly soon as
%  we don't understand the LV2 code very well.  There are valid reasons for
%  wanting different sizes for these buffers. The internal buffer size as well
%  as affecting latency and CPU load also alters the sound in quite subtle
%  ways. It particularly affects the behaviour of filters.  One day we may be
%  able to stop this happening, but in the mean time we have to live with it.
%  For our purposes, we find a buffer size of 128 or 256 is best.

%-------------------------------------------------------------------------------
% vim: ts=3 sw=3 et ft=tex
%-------------------------------------------------------------------------------
