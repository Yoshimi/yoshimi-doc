%-------------------------------------------------------------------------------
% yum_configuration
%-------------------------------------------------------------------------------
%
% \file        yum_configuration.tex
% \library     Documents
% \author      Chris Ahlstrom
% \date        2016-03-07
% \update      2016-06-05
% \version     $Revision$
% \license     $XPC_GPL_LICENSE$
%
%     Provides descriptions of the configuration files.
%     Not yet part of the document.
%
%-------------------------------------------------------------------------------

\section{Configuration Files}
\label{sec:configuration}

   Let's cover the configuration files, which have expanded in utility in
   recent versions of \textsl{Yoshimi}.
   Understanding these configuration file makes it easier to 
   use \textsl{Yoshimi}.

   As with most applications, \textsl{Yoshimi} and \textsl{ZynAddSubFX}
   allow for one to save one's work and reload it.
   \textsl{Yoshimi} has a number of different files that make up the current
   configuration.
   Together, they make up the concept of a \textsl{patch set} (also called a
   \textsl{patchset}).
   Sometimes one will see reference to a "session", but that term is too easy
   to confuse with the "session" in "JACK session manager".
   Here are the file extensions used for saving the \textsl{Yoshimi}
   patch-set data:

   Here is a summary of the files.  Please note that the names all start with
   \texttt{yoshimi}.  For example, \texttt{.banks} is really
   \texttt{yoshimi.banks}.

   \begin{itemize}
      \item \texttt{.banks}
         \index{.banks}
         Contains information on the accessible instrument banks, and
         information to translate between bank directory names and bank ID
         values.
      \item \texttt{.config}
         \index{.config}
         Contains the setup information configured in the
         \textbf{Yoshimi / Settings} dialog.
      \item \texttt{.history}
         \index{.history}
         Recent patch sets are now stored in the \texttt{.history} file.
         (Does this include command history?)
      \item \texttt{.instance(n)}
         \index{.instance(n)}
         Contains the current root/bank, MIDI settings, and preferred engines.
         (This file does not yet exist, but is a planned future feature.)
      \item \texttt{.state}
         \index{.state}
         Contains the state information needed to
         duplicate a \textsl{Yoshimi} session that was saved.
      \item \texttt{.windows}
         \index{.windows}
         Contains the current layout of windows for reinstantiation at the next
         startup of \textsl{Yoshimi}.
         If there is no such file
         (\texttt{\textasciitilde/.config/yoshimi/yoshimi.window}) at
         \textsl{Yoshimi}, then the keyboard is also opened, alongside the main
         window, as a help to those new to \textsl{Yoshimi}.
         And of course that state will be saved, if present, when
         \textsl{Yoshimi} exits.
      \item \texttt{.xiz}
         \index{.xiz}
         An Instrument file.
      \item \texttt{.xmz}
         \index{.xmz}
         A full \textsl{Yoshimi} configuration; everything.
         This file is also called a \textsl{parameter set}.
      \item \texttt{.xpz}
         \index{.xpz}
         Presets.
         A preset is a \textsl{Yoshimi} sub-setting file.
      \item \texttt{.xsz}
         \index{.xsz}
         Scale Settings.
      \item \texttt{.xvy}
         \index{.xvy}
         Vector settings. The extension stands for "Xml Vector Yoshimi".
         It will eventually be integrated with the saved states.
         For a good example, see \sectionref{subsection:vector_command_line}.
         One day, vector settings will also be part of the state save, and
         possibly patch sets too.
   \end{itemize}

   The entire config set should then be (ignoring the prepended
   \texttt{yoshimi}):

   \begin{itemize}
      \item \texttt{.config}
      \item \texttt{.instance[n]} (future)
      \item \texttt{.windows}
      \item \texttt{.history}
      \item \texttt{.banks} (this is currently per instance)
   \end{itemize}

   The \texttt{.windows} file is specific to the GUI, so doesn't figure in this
   scheme at all, but it is created or saved when one exits
   \textsl{Yoshimi}.

   In the file-save dialogs, the file extension is determined by the type of
   file being saved, and it doesn't matter if one enters the extension
   explicity, or not. If it's missing, or it is the wrong one, it will be
   replaced. This is actually true of almost all file saves, and has been for
   quite some time now.

   For vectors (in common with external instruments and patch sets), it's up to
   the user as to where to save. The file filter generally defaults to the
   either the user home directory, or if \textsl{Yoshimi} was launched from
   userland, it's the directory it launched from. Then it's the normal
   file browser selection.  Once saved,
   \textbf{Vectors / Options / Recent} is your friend.

\subsection{Configuration Files / Patch Set}
\label{subsec:configuration_patch_set}

   \index{.xmz}
   \index{patch set}
   \index{file!patch set}
   A patch set is basically a group of instruments related simply by the user
   wanting to have them all loaded at once into \textsl{Yoshimi}.  A patch set
   is stored in a \texttt{.xmz} file.  A patch set is akin to a preset, in that
   it stores a combination of items, that took awhile to set up, for easy
   retrieval later.

   Patch sets are not the full configuration. They carry \textsl{most} of it,
   including almost all of the dynamic settings, but they don't contain the
   configuration settings that \texttt{.state} does.  The patch set format is
   either XML or compressed XML, as explained elsewhere.  The
   \textbf{Patch Sets / Save External...} menu entry saves files with
   the \texttt{.xmz} extension.
   (This file used to be called \textsl{parameter
   set}, but that name is no longer used in \textsl{Yoshimi}.)

   One of the simplest ways to save one's work is to save the entire
   \textsl{Yoshimi} configuration.
   This saving can be done through the \textbf{Patch Sets} menu
   (the \textbf{File} menu in \textsl{ZynAddSubFX}),
   and will result in the creation of
   a \texttt{.xmz} file. Once created, this file will hold the settings for
   all settings within that setup, such as microtonal tunings, all
   patches, system effects, insertion effects, etc.
   See \sectionref{paragraph:menu_yoshimi_settings_main_settings}.

   In many cases saving everything in a part is not what is desired.
   Saving a patch later on in an editing session is one such example.
   In order to save a patch, one can either save it from the
   \textbf{Instruments} menu, or through the \textbf{Bank} window.

\subsection{Configuration Files / Config}
\label{subsec:configuration_config}

   \index{.config}
   \index{config}
   \index{file!config}
   Often, one will see the extension \texttt{.config} used in the
   \texttt{\$HOME/.config/yoshimi} directory.  This file once contained
   information to translate between bank directory names and bank ID
   values.  In recent versions of \textsl{Yoshimi}, this file is much
   reduced in size, and its "doctype" is no longer "ZynAddSubFX".

   The \texttt{.config} file is always going to be specific to one machine and
   working modes, so no one will ever want to copy it across even to another
   \textsl{Yoshimi} environment.  Recent patch sets are now no longer stored in
   the main \texttt{.config} file, but in a new \texttt{.history} file.  The
   \texttt{.config} file is now a much reduced common settings -- interfaces,
   sample rate -- file.  It is a single file that every instance can read, but
   only the first one can write.
   
   \texttt{.config} hasn't yet been separated
   from \texttt{.instance(n)} and all files are still per instance, as the
   \textsl{Yoshimi} team haven't had time to work out exactly how to manage
   common files and memory locations for those that should be shared.  The
   \texttt{.config} file is saved only when the user explicitly calls for it to
   be saved.
   \textsl{Yoshimi} will still mention its absense:

   \begin{verbatim}
      $ yoshimi -a -A
      Yoshimi is starting
      ConfigFile /home/ahlstrom/.config/yoshimi/yoshimi.config not found, will
         use default settings ...
   \end{verbatim}

   The \texttt{.config} file will be readable by all instances of
   \textsl{Yoshimi}, but writeable only by the main instance. The relevant
   controls will be greyed out in the other instances.  The \texttt{.config} and
   \texttt{.banks} data now reside in separate configuration files.  The banks
   file is saved every time there is a normal exit, so the last-used root and
   bank IDs will always match what that instance thinks is there.  Conversely,
   the main \texttt{.config} file \textsl{doesn't} get saved when one starts a
   new (unkown) instance of \textsl{Yoshimi}, but the config-changed flag is
   set, so one has control over whether any settings are saved.  So now, if
   anything goes wrong with the config files they won't corrupt one's carefully
   organised bank files, and vice-versa.

\subsection{Configuration Files / State}
\label{subsec:configuration_state}

   \index{.state}
   \index{state}
   \index{file!state}
   Sometimes one will see the extension \texttt{.state} used in the
   \texttt{\$HOME/.config/yoshimi} directory.  These files contain a lot more
   information, that needed to duplicate a \textsl{Yoshimi} session that was
   saved.  Seems to be a superset of an \texttt{.xmz} file, saving everything.
   This is a facility that is peculiar to \textsl{Yoshimi}.
   The state file is accessed from the \textbf{State} menu item in the main
   window.
   Its default name is
   \texttt{\textasciitilde/.config/yoshimi/yoshimi.state}.

   The \textsl{Yoshimi} 'state' file consists of the entire setup, from basic
   configuration settings to currently-loaded instrument sets.
   However, upon investigating some JACK
   session managers, it looks like they don't want (or can't use) most of the
   configuration information because they are expecting to be able to change
   the state in \textsl{running} instances.

   If and when \textsl{Yoshimi} gets around to splitting the 'instance' data
   from the main configuration, then a lot of this session issue can be resolved
   buy saving only the 'true' configuration locally, and to the state save.
   However, the 'instance' data will include things like ALSA/JACK settings.
   Currently we can't change these live (although it would be nice if we could),
   but would anyone want to do so from a JACK session manager?

\subsection{Configuration Files / Instrument}
\label{subsec:configuration_instrument}

   \index{.xiz}
   \index{instrument}
   \index{file!instrument}
   An Instrument.  These files can have two formats, compressed and
   uncompressed.
   With the \textbf{Instrument} menu, one can just save the file to any
   given location with the \texttt{.xiz} extension.

\subsection{Configuration Files / Scale}
\label{subsec:configuration_scale}

   \index{.xsz}
   \index{scale}
   \index{file!scale}
   Scale Settings.  These files store microtonal settings that \textsl{Yoshimi}
   can use to produce non-standard musical scales.  Recent scales settings are
   saved and recorded.

\subsection{Configuration Files / Presets}
\label{subsec:configuration_preset}

   \index{.xpz}
   \index{preset}
   \index{file!preset}
   Have a favorite setting for an envelope, or a difficult-to-reproduce
   oscillator? Then presets are for you! Presets allow for one to save the
   settings for any of the components which support copy/paste operations.
   This is done with preset files (\texttt{.xpz}), which get stored in the
   folders indicated by \textsl{Paths / Preset Dirs...}.
   The key thing about using presets is that one must first
   specify a presets directory!  Otherwise, who knows where they go?
   A good choice for a preset directory is
   \texttt{\textasciitilde/.config/yoshimi/presets}.

   In \textsl{Yoshimi}, a
   \textsl{preset} is any collection of settings that can be saved to the
   clipboard or to a file, for later loading elsewhere.

   A preset is canned version of a \textsl{Yoshimi} sub-setting.  Presets can be
   copied and pasted using the \textbf{C} and \textbf{P} user-interface buttons
   associated with many of the \textsl{Yoshimi} dialog windows.  They make it
   easy to save portions of the current settings for later use.  For example,
   resonance settings can be saved.

   The naming convention for a preset file is
   \texttt{presetname.presettype.xpz}, where
   \textsl{presename} is the name one types into the \textbf{Copy to Preset}
   name field, \textsl{presettype} is the name that appears in the
   \textbf{Type} field, and \textsl{xpz} is the file-extension for compressed
   XML preset files.

% \subsection{Configuration Files / Patch Sets}
% \label{subsec:configuration_patch_sets}

\subsection{Configuration Files / Instance}
\label{subsec:configuration_instance}

   \index{.instance(n)}
   \index{instance}
   \index{file!instance}
   A new feature of the \textsl{Yoshimi} configuration.
   It contains the current root/bank, MIDI settings, and preferred engines.
   These instance files are totally independent files, distinguished by a number
   in the file-name.

\subsection{Configuration Files / History}
\label{subsec:configuration_history}

   \index{.history}
   \index{history}
   \index{file!history}
   A new feature of the \textsl{Yoshimi} configuration.
   Recent patch sets are now stored in the \texttt{.history} file.
   For example, if the \textbf{XML Compression} option is set to 0, and one
   exits \textsl{Yoshimi}, then the file
   \texttt{\textasciitilde/.config/yoshimi/yoshimi.history} might
   contain the following items (ignoring the XML markup):

   \begin{verbatim}
		/home/me/yoshimi-cookbook/sequencer64/b4uacuse/yoshimi-b4uacuse-gm.state
		/home/me/sequencer64/contrib/yoshimi/horse.state
   \end{verbatim}

   \texttt{.history} is a single file that every instance can read and write.
   The \texttt{.history} file is saved only upon a normal exit, as it is
   comparatively unimportant.

   The history is a single buffer and file, readable and writeable by all
   instances. This is actually quite interesting as there can never be a
   conflict.  It is impossible to have two browser lists open at the same time
   (try it!) and the lists are always rebuilt from memory every time they are
   opened. Similarly, the histories are added too every time a new recognised
   file is loaded or saved and one can't physically do two at the same time --
   even if one could it would simply mean that one very briefly waited for the
   other, which is not an issue as they are not in the realtime thread.

   Now, there is also another "history" file apparently created by
   \textsl{Yoshimi}: \texttt{\textasciitilde/.yoshimi\_history}.
   This file seems to be a history of commands typed into the command-line
   interface of \textsl{Yoshimi}.  TO BE DETERMINED.

\subsection{Configuration Files / Banks}
\label{subsec:configuration_banks}

   \index{.banks}
   \index{banks}
   \index{file!banks}
   A new feature of the \textsl{Yoshimi} configuration.  Currently each
   \textsl{Yoshimi} instance takes its own copy of the actual files as it starts
   up.  However, they can all save, delete, or rename the actual files without
   talking to the other instances, so one can move a file in one instance, and
   then try (and fail) to access it from another.

   With the \textbf{Banks} menu, one can assign a patch to a given slot with
   a bank.  This instrument will remain in that slot for future use until it is
   deleted. To see the physical location of the \texttt{.xiz} file, one
   should check the
   \textbf{Yoshimi / Settings / Banks / Root Dirs}
   (\textsl{File→Settings→Bank\_Root\_Dirs}) window to see the paths for
   banks.

   At startup, after all the configuration is complete, the banks are loaded and
   installed.  On a per instance basis, the first thing this process does is
   look for a \texttt{yoshimi(-n).banks} file, if it can't, find that it then
   hunts for a \texttt{yoshimi(-n).config} file, and if that fails it does a
   rescan for banks. In this way it should be completely backward compatible
   with any previous config files.

   The \texttt{.banks} file is \textsl{saved} every time roots, banks, or
   instruments are changed, and again on a normal exit to catch the current
   root and bank (which don't otherwise trigger a save).  This allows the
   last-used root and bank IDs to always match what that instance thinks is
   there.
   Note that one needs to have write permissions to add instruments to the
   bank.
   
   Banks are more thoroughly described in
   \sectionref{subsec:concepts_banks_and_roots}.

\subsection{Configuration Files / Windows}
\label{subsec:configuration_windows}

   No, this term isn't a reference to "that other operating system".

   \index{.windows}
   \index{windows}
   \index{file!windows}
   A new feature of the \textsl{Yoshimi} configuration.  It saves the current
   layout of windows for reinstantiation at the next startup of
   \textsl{Yoshimi}.

\subsection{Configuration Files / Format}
\label{subsec:configuration_file_format}

   The Unix \texttt{file} command indicates that the XML files are one of
   two types:

   \begin{itemize}
      \item \textsl{exported SGML document, ASCII text}.
         These files are unindented XML data with an encoding of UTF-8 and
         a DOCTYPE of "ZynAddSubFX-data".
      \item \textsl{gzip compressed data, from Unix}.
         These files can be renamed to end in ".gz", and then run through
         the \texttt{gunzip} program to yield the XML file (but without an
         \texttt{.xml} extension).
   \end{itemize}

   The format depends on the "XML compression level" option discussed in
   \sectionref{paragraph:menu_yoshimi_settings_main_settings}.

   \index{saving settings}
   Saving settings or not:
   If one changes settings, and closes without saving, that means the settings
   remain in place only for the current session. If one has changed anything,
   when one closes \textsl{Yoshimi}, one will be given a second chance to
   save them. If one responds 'No',  the next time \textsl{Yoshimi} starts,
   the old settings will be restored.  An 'undo' feature would get pretty
   crazy very quickly.

   In the \textbf{Settings} window, \textbf{Save Settings}
   refers to the entire window, not just individual tabs. The close buttons are
   actually outside the frame of the tabs.

   \textbf{Close without saving} doesn't mean revert to previous settings; it
   means to use the changes, but don't immediately store them to the
   filesystem.

   In general, the contents are structured a lot like the
   user-interface elements that are used to set them.

%-------------------------------------------------------------------------------
% vim: ts=3 sw=3 et ft=tex
%-------------------------------------------------------------------------------
