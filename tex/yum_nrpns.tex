%-------------------------------------------------------------------------------
% yum_nrpns
%-------------------------------------------------------------------------------
%
% \file        yum_nrpns.tex
% \library     Documents
% \author      Chris Ahlstrom
% \date        2015-06-26
% \update      2019-12-08
% \version     $Revision$
% \license     $XPC_GPL_LICENSE$
%
%     Provides the concepts NRPNs and effects.
%
%-------------------------------------------------------------------------------

\section{Non-Registered Parameter Numbers}
\label{sec:nrpns}

   \textsl{Yoshimi} implements System and Insertion effects control in a
   manner compatible with \textsl{ZynAddSubFX}. As with all
   \textsl{Yoshimi}'s NRPNs, the controls can be sent on any MIDI channel.

\subsection{NRPN / Basics}
\label{subsection:nrpns_midi_nrpn_basics}

   NRPN stands for "Non Registered Parameters Number".
   NRPNs can control all System and Insertion effect parameters.
   Using NRPNs, \textsl{Yoshimi} can directly set some part values
   regardless of what channel that part is connected to.  For example, one
   may change the reverb time when playing on a keyboard, or
   change the flanger's LFO frequency.
%  One can disable the NRPN receiving by deselecting the "NRPN" checkbox
%  from the main window (near "Master Keyshift" counter).
   The controls can be sent on any MIDI channel
   (the MIDI channels numbers are ignored).

   The parameters are:

   \begin{itemize}
      \item \textbf{NRPN MSB}
      (coarse) (99 or 0x63) sets the system/insertion effects
      (4 for system effects or 8 for insertion effects).
      We abbreviate this value as \texttt{Nhigh}.
      \item \textbf{NRPN LSB}
      (fine) (98 or 0x62) sets the number of the effect (first
      effect is 0).
      We abbreviate this value as \texttt{Nlow}.
      \item \textbf{Data entry MSB}
      (coarse) (6) sets the parameter number of effect to
      change (see below).
      We abbreviate this value as \texttt{Dhigh}.
      \item \textbf{Data entry LSB}
      (fine) (26) sets the parameter of the effect.
      We abbreviate this value as \texttt{Dlow}.
   \end{itemize}

   One must send NRPN coarse/fine before sending Data entry coarse/fine.  If
   the effect/parameter doesn't exist or is set to none, then the NRPN is
   ignored.

   Also note that when a parameter, such as channel or program is to be modified
   by an NRPN, the value is specified as \textsl{raw} data, meaning the values
   start at \textbf{0}, not \textbf{1}.

   It's advisable to set NRPN MSB before LSB. However, once MSB has
   been set one can set a chain of LSBs if they share the same MSB.
   The data CCs associated with these are 6 for MSB and 38 for LSB.
   Only when an NRPN has been established can the data values be entered
   (they will be ignored otherwise).
   If a supported control is identified, these data values will be stored
   locally (if needed) so that other NRPNs can be set.
   Whenever either byte of the NRPN is changed, the data values will be
   cleared (but stored settings will not be affected).
   If either NRPN byte is set to 127, all data values are ignored again.

   In \textsl{Yoshimi}, NRPNs are not themselves channel-sensitive, but the
   final results will often be sent to whichever is the current channel.
   \textsl{Yoshimi} also supports the curious 14-bit NRPNs, but this shouldn't
   be noticeable to the user.
   (In these notes, where practical we also list the 14 bit values in square
   brackets.)

   In order to deal with some
   variations in the way sequencers present NRPNs generally, if a complete
   NRPN is set
   (i.e. \texttt{Nhigh}, \texttt{Nlow}, \texttt{Dhigh}, \texttt{Dlow}),
   then the data bytes can be in
   either order, but must follow \texttt{Nhigh} and \texttt{Nlow}.

   After this, for running values, once
   \texttt{Dhigh} and \texttt{Dlow} have been set, if one
   changes either of these values, the other value will be assumed.
   For example, starting with \texttt{Dhigh} = 6 and \texttt{Dlow} = 20:

   Change \texttt{Dlow} to 15, and \textsl{Yoshimi} will regard this as a
   command \texttt{Dhigh} 6 + \texttt{Dlow} 16.
   Alternatively, change
   \texttt{Dhigh} to 2, and \textsl{Yoshimi} will regard this as a
   command \texttt{Dhigh} 2 + \texttt{Dlow} 20.
   This can be useful but may have unintended consequences!
   If in doubt; change either of the NRPN bytes and both data bytes will be
   cleared.

   Additionally there is CC 96 for data increment, and CC 97 for data
   decrement.

   Data increment and decrement operation enables one to directly change the
   data LSB by deltas between 0 and 63.
   To change the MSB, add 64 to cover the same
   range. Setting 0 might seem pointless, but it gives an alternative way
   to make an initial setting if one's sequencer doesn't play nice.

   Although data increment and decrement are only active if a valid NRPN has
   been set, they are otherwise quite independent single CCs.  For example:

   \begin{verbatim}
      Start  Value      Command value   Result
      -----  -----      --------------  ------
      LSB     5           inc 20         25
      MSB     7           inc 68         11
      LSB     128(off)    inc 1          1
      MSB     126         dec 74         116
      MSB     128(off)    dec 65         127
   \end{verbatim}

   A small example (all values in this example are hex):

   \begin{verbatim}
       B0 63 08 // Select the insertion effects
       B0 62 01 // Select the second effect (remember: the first is 00 and not 01)
       B0 06 00 // Select the effect parameter 00
       B0 26 7F // Change the parameter of effect to the value 7F (127)
   \end{verbatim}

\subsection{CC and NRPN Quick Reference}
\label{subsection:cc_nrpn_quick_ref}
   The table \tableref{table:CCandNRPNreference}, is a simple list of all the
   Continuous Controllers (CCs) and Non Registered Parameter Numbers (NRPNs)
   recognised by Yoshimi.
   For more CC details see \sectionref{subsubsec:concepts_midi_messages}, and for NRPN details \sectionref{sec:nrpns}.

\index{new!CC and NRPN Quick Reference}
      \begin{table}[H]
      \centering
      \caption{CC and NRPN Quick Reference}
      \label{table:CCandNRPNreference}
      \begin{tabular}{c c l}
    \textbf{CC} & & \textbf{Function} \\
       Hex (Dec) \\
       00 (0) & & Bank MSB\\
       01 (1) & & Modulation Wheel\\
       02 (2) & & Breath\\
       06 (6) & & NRPN Data MSB\\
       07 (7) & & Volume\\
       0A (10) & & Panning\\
       0B (11) & & Expression\\
       20 (32) & & Bank LSB\\
       26 (38) & & NRPN Data LSB \\
       40 (64) & & Sustain Pedal \\
       41 (65) & & Prtamento \\
       44 (68) & & Legato Pedal \\
       47 (71) & & Filter Q \\
       4A (74) & & Filter Cutoff \\
       4B (75) & & Bandwidth \\
       4C (76) & & FM Amplitude \\
       4D (77) & & Resonance Centre Frequency\\
       4E (78) & & Resonance Bandwidth \\
       60 (96) & & NRPN Data Increment \\
       61 (97) & & NRPN Data Decrement \\
       62 (98) & & NRPN LSB \\
       63 (99) & & NRPN MSB \\
       78 (120) & & All Sound Off \\
       79 (121) & & Reset All Controllers \\
       7B (123) & & All Notes Off\\
       \\
    \textbf{NRPN MSB} & \textbf{NRPN LSB} & \textbf{Function} \\
        Hex (Dec) & Hex (Dec)\\
        04 (4) & 00-04 (0-4)& System Effect number \\
        08 (8) & 00-08 (0-8)& Insertion Effect number \\
        40 (64) & 00 (0) & Direct part control \\
        40 (64) & 01 (1) & Vector Control \\
        40 (64) & 02 (2) & System Settings \\
        60 (96) & 00-05 (0-5) & Load Numbered File from List \\
        \\
    \textbf{ShortForm} &  LSB is value & No data bytes needed \\
        41 (65) & 00-05 (0-5) & Set Solo Switch Type \\
        42 (66) & 00-77 (0-119) & Set Solo Switch Incomming Controller \\
        44 (68) & 44 (68) & Perform Normal Shutdown, exit value 0 \\
        44 (68) & 45 (69) & Perform shutdown without messages, exit value 16 \\
      \end{tabular}
   \end{table}

\subsection{Effects NRPN values}
\label{subsection:effects_nrpn_values}
   \textbf{WARNING}:
   Changing of some of the effect parameters produces clicks when sounds
   passes thru these effects.  We advise one to change only when the sound
   volume that passes through the effect is very low (or silence).  Some
   parameters produce clicks when they are changed rapidly.

   Here are the effects parameter numbers (for Data entry, coarse).
   The parameters that produce clicks are written in \textcolor{red}{red}
   and have (AC) after their entry (always clicks).
   The parameter that produces clicks only when they are changed fast are
   written in \textcolor{blue}{blue} and have a (FC) after the entry (Fast
   Clicks).
   Most parameters have the range from 0 to 127.  All numbers are specified re 0,
   not 1.  For example, channel numbers range from 0 to 15, not 1 to 16, when
   used in an NRPN setting.
   When parameters have another range, it is written as "(low..high)".

   Here are the basic formats:

   \begin{enumerate}
      \item Send NRPN:
      \begin{itemize}
         \item MSB = 64 (same as for vectors)
         \item LSB = 0
      \end{itemize}
      \item Send Data MSB (6); all value ranges start from zero, not 1.
      \begin{itemize}
         \item 0 : data LSB = part number
         \item 1 : data LSB = program number
         \item 2 : data LSB = controller number
         \item 3 : data LSB = controller value
         \item 4 : data LSB = part's channel number (16 to 31 allows only
            Note Off for this part, while numbers 32 or above disconnects
            the part from all channel message)
         \item 5 : data LSB = part's audio destination, one of
                    1 = main L\&R;
                    2 = direct L\&R;
                    3 = both;
                    all other values are ignored
         \item 7 : data LSB = main volume
         \item 35 (0x23) : data LSB = controller LSB value (not yet implemented)
         \item 39 (0x27) : data LSB = main volume LSB (not yet implemented)
         \item 64 (0x40) : data LSB = key shift value (64 = no shift)
         \item Other values are currently ignored.
      \end{itemize}
   \end{enumerate}

   Other values are currently ignored by \textsl{Yoshimi}.

\subsection{NRPN / Effects Control}
\label{subsection:nrpns_midi_nrpn_effects_control}

\paragraph{Reverb}

   \begin{itemize}
      \item \textcolor{blue}{00 - Volume or Dry/Wet (FC)}
      \item \textcolor{blue}{01 - Pan (FC)}
      \item 02 - Reverb Time
      \item \textcolor{blue}{03 - Initial Delay (FC)}
      \item 04 - Initial Delay Feedback
      \item \textcolor{magenta}{05 - reserved}
      \item \textcolor{magenta}{06 - reserved}
      \item 07 - Low Pass
      \item 08 - High Pass
      \item 09 - High Frequency Damping (64..127) 64=no damping
      \item \textcolor{red}{10 - Reverb Type (0..1) 0-Random, 1-Freeverb (AC)}
      \item \textcolor{red}{11 - Room Size (AC)}
   \end{itemize}

\paragraph{Echo}

   \begin{itemize}
      \item \textcolor{blue}{00 - Volume or Dry/Wet (FC)}
      \item \textcolor{blue}{01 - Pan (FC)}
      \item \textcolor{red}{02 - Delay (AC)}
      \item \textcolor{red}{03 - Delay between left and right (AC)}
      \item \textcolor{blue}{04 - Left/Right Crossing (FC)}
      \item 05 - Feedback
      \item 06 - High Frequency Damp
   \end{itemize}

\paragraph{Chorus}

   \begin{itemize}
      \item \textcolor{blue}{00 - Volume or Dry/Wet (FC)}
      \item \textcolor{blue}{01 - Pan (FC)}
      \item 02 - LFO Frequency
      \item 03 - LFO Randomness
      \item 04 - LFO Type (0..1)
      \item 05 - LFO Stereo Difference
      \item 06 - LFO Depth
      \item 07 - Delay
      \item 08 - Feedback
      \item \textcolor{blue}{09 - Left/Right Crossing (FC)}
      \item \textcolor{magenta}{10 - reserved}
      \item \textcolor{red}{11 - Mode (0..1) (0=add, 1=subtract) (AC)}
   \end{itemize}

\paragraph{Phaser}

   \begin{itemize}
      \item \textcolor{blue}{00 - Volume or Dry/Wet (FC)}
      \item \textcolor{blue}{01 - Pan (FC)}
      \item 02 - LFO Frequency
      \item 03 - LFO Randomness
      \item 04 - LFO Type (0..1)
      \item 05 - LFO Stereo Difference
      \item 06 - LFO Depth
      \item 07 - Feedback
      \item \textcolor{red}{08 - Number of stages (0..11) (AC)}
      \item \textcolor{blue}{09 - Let/Right Crossing (FC)}
      \item \textcolor{red}{10 - Mode (0..1) (0=add, 1=subtract) (AC)}
      \item 11 - Phase
   \end{itemize}

\paragraph{AlienWah}

   \begin{itemize}
      \item \textcolor{blue}{00 - Volume or Dry/Wet (FC)}
      \item \textcolor{blue}{01 - Pan (FC)}
      \item 02 - LFO Frequency
      \item 03 - LFO Randomness
      \item 04 - LFO Type (0..1)
      \item 05 - LFO Stereo Difference
      \item 06 - LFO Depth
      \item 07 - Feedback
      \item 08 - Delay (0..100)
      \item \textcolor{blue}{09 - Left/Right Crossing (FC)}
      \item 10 - Phase
   \end{itemize}

\paragraph{Distortion}

   \begin{itemize}
      \item \textcolor{blue}{00 - Volume or Dry/Wet (FC)}
      \item \textcolor{blue}{01 - Pan (FC)}
      \item 02 - Left/Right Crossing
      \item \textcolor{blue}{03 - Drive (FC)}
      \item \textcolor{blue}{04 - Level (FC)}
      \item 05 - Type (0..11)
      \item 06 - Invert the signal (negate) (0..1)
      \item 07 - Low Pass
      \item 08 - High Pass
      \item 09 - Mode (0..1) (0=mono,1=stereo)
   \end{itemize}

\paragraph{EQ}

   \begin{itemize}
      \item \textcolor{blue}{00 - Gain (FC)}
   \end{itemize}

   All other settings of the EQ are shown in a different way.
   The N represent the band ("B." setting in the UI) and the first band is 0
   (and not 1), like it is shown in the UI.
   Change the "N" with the band one likes.
   If one wants to change a band that doesn't exist, the NRPN will be ignored.

   \begin{itemize}
      \item \textcolor{red}{10+N*5 - Change the mode of the filter (0..9) (AC)}
      \item 11+N*5 - Band's filter frequency
      \item 12+N*5 - Band's filter gain
      \item 13+N*5 - Band's filter Q (bandwidth or resonance)
      \item \textcolor{magenta}{14+N*5 - reserved}
   \end{itemize}

   Example of setting the gain on the second band in the EQ module:

   \begin{itemize}
      \item The bands start counting from 0, so the second band is
         1 =\textless N=1.
      \item The formula is 12+N*5 =\textless 12+1*5=17, so the number of effect
         parameter (for Data entry coarse) is 17.
   \end{itemize}

\paragraph{DynFilter}

   \begin{itemize}
      \item 0 - Volume
      \item 1 - Pan
      \item 2 - LFO Frequency
      \item 3 - LFO Randomness
      \item 4 - LFO Type
      \item 5 - LFO Stereo Difference
      \item 6 - LFO Depth
      \item 7 - Filter Amplitude
      \item 8 - Filter Amplitude Rate Change
      \item 9 - Invert the signal (negate) (0..1)
   \end{itemize}

   Click behaviour of DynFilter has not yet been tested.

\paragraph{Yoshimi Extensions}

   If the Data MSB bit 6 is set (64) then Data LSB sets the effect type
   instead of a parameter number.  This must be set before making a parameter
   change.

   \begin{itemize}
      \item 0 - Reverb
      \item 1 - Echo
      \item 2 - Chorus
      \item 3 - Phaser
      \item 4 - AlienWah
      \item 5 - Distortion
      \item 6 - EQ
      \item 7 - DynFilter
   \end{itemize}


   For System effects, if the Data MSB bit 5 (32) is set
   \textsl{and} Data MSB bit 6 (64) is set (i.e. a combined value of 96),
   then the command sets how much of one system effect is sent to another.
   The range of values depend on which is the sending effect. It can't be sent to itself, or a lower numbered one.

   The MSB would be from 96 to 98, representing effect 2 to 4.

   The Data LSB is the amount to actually send.

   For Insert effects, if the Data MSB bit 5 (32) is set
   \textsl{and} Data MSB bit 6 (64) is set (i.e. a combined value of 96),
   then Data LSB sets the destination part number.
   127 is off and 126 is the Master Output.
   A complete example:

   \begin{itemize}
      \item 99 -   8 \textasciitilde insert effects
      \item 98 -   3 \textasciitilde number 4 (as displayed)
      \item 6 -  32 \textasciitilde set destination
      \item 38 - 126 \textasciitilde Master Out
      \item 99 -   8  *
      \item 98 -   3  *
      \item 6 -  64 \textasciitilde change effect
      \item 38 -   4 \textasciitilde Alienwah
      \item 99 -   8  *
      \item 98 -   3  *
      \item 6 -   0 \textasciitilde Dry/Wet
      \item 38 -  30 \textasciitilde value
   \end{itemize}

   Notes (*): these repeats are not needed for \textsl{Yoshimi},
   but some sequencers are unhappy without them.
   Changing just a parameter on an existing system effect:

   \begin{itemize}
      \item 99 -   4 \textasciitilde system effects
      \item 98 -   0 \textasciitilde the first effect
      \item 6 -   1 \textasciitilde Pan
      \item 38 -  75 \textasciitilde value
   \end{itemize}


\subsection{NRPN / Direct Part Controls}
\label{subsection:nrpns_direct_part_control}
   All of these controls are assigned to a specific part, regardless of what channel number it is on and in most cases doesn't even require the part to be enabled. They do however need to be in the current range available. i.e if the range is 32 parts, one can only set numbers 0 to 31.

   Data MSB is the control to be managed, and Data LSB is the value to set.

   \begin{itemize}
      \item 0 \hspace{8pt}Part number (this must be set first)
      \item 1 \hspace{8pt}Program Change
      \item 2 \hspace{8pt}Controller to change (0 - 119)
      \item 3 \hspace{8pt}Controller value
      \item 4 \hspace{8pt}Channel number
      \begin{enumerate}[label=\alph*)]
          \item 0 - 15 select for all messages
          \item 6 - 31 select for note off only
          \item 32 - 47 mute
      \end{enumerate}
      \item 5 * Audio destination (0 - 2) main/part/both
      \item 8 \hspace{8pt}Send part output the System effect 1
      \item 9 \hspace{8pt}Send part output the System effect 2
      \item 10 \hspace{8pt}Send part output the System effect 3
      \item 11 \hspace{8pt}Send part output the System effect 4
      \item 64 * Key shift (28 - 100) gives -36 to +36
   \end{itemize}
   Note (*): requires the part to be enabled.
\subsection{NRPN / Dynamic System Settings}
\label{subsection:nrpns_dynamic_system_settings}

   Almost all dynamic setup (i.e. that doesn't require a restart) can now be
   done via NRPNs, so a MIDI file can manage \textsl{Yoshimi} starting from a
   pretty random state, and set up important features like Bank and Program
   Change behaviour and the number of available parts.

   In parallel with this setup, there is a command to list all of these
   settings. One can also list the available bank roots, the banks in any
   root, and instruments in any bank, along with their numeric IDs. These IDs
   can then be used with normal MIDI CCs to get exactly the instrument you
   want at any time.

   This arrangement looks positively steam-punk, but is actually very easy to
   use, requiring only a command line interface and any utility that can send
   MIDI CCs. NRPNs aren't special. They are simply a specific pattern of CCs.
   \textsl{Yoshimi}'s implementation is very forgiving, doesn't mind if you
   stop halfway through (will just get on with other things while it waits),
   and will report exactly what it is doing.  So ...

   ... If \textsl{Yoshimi} has been started from the command line (but not
   necessarily in the no-GUI mode), all of the system settings that don't
   require a restart can now be viewed by sending the appropriate NRPN. Most
   of them can also be changed in this way.
   To access this functionality, set NRPN MSB (CC 99) to 64 and NRPN LSB (CC
   98) to 2 (8130).

   After that send the following DATA values. Commands with LSB x don't
   actually use DATA LSB, but one still needs to send it (unless it has
   already been set by a previous command in this control group).

   \begin{table}[H]
      \centering
      \caption{Dynamic System Commands}
      \label{table:dynamic_system_commands}
      \begin{tabular}{r l l}

\textbf{DATA MSB} & \textbf{DATA LSB} & \textbf{Setting} \\

  2 & LSB key       & Set master key shift, 52\textless=key\textless=76 (-36 to +36) \\
  7 & LSB volume    & Set master Volume 'volume' \\
  64-79 & LSB key   & Set channel-based (MSB-64) key shift, key-64 (-36 to +36) \\
  80    & root      & Set which CC will control Root path change (\textgreater119 disables) \\
  81    & bank      & Set wich CC will control Bank change (\textgreater119 disables) \\
  82    & \textgreater63       & Enable Program change otherwise disable \\
  83    & \textgreater63       & Enable activation of part when program changed \\
  84    & extended  & Set CC control Extended program change (\textgreater119 disables) \\
  85    & parts     & Set number of available parts (16, 32 or 64) \\
  86    & x         & Save all dynamic settings \\
      \end{tabular}
   \end{table}

\subsection{NRPN / Load From History Index}
\label{subsection:nrpns_load_from_history_index}
\index{new!NRPN History Loading}
   Since \textsl{Yoshimi} V 1.6.0 there is a method of loading external
   instruments, patch sets, etc. from their history lists. This is only
   practical if the relative history has been locked in settings.

   For all of these the NRPN MSB is 96 (0x60), and the LSB determines which
   list type is to be handled.

   \begin{table}[H]
      \centering
      \caption{History Load Commands}
      \label{table:history_load_commands}
      \begin{tabular}{c l c l c l}
\textbf{NRPN LSB} & \textbf{Type} & \textbf{DATA MSB} & \textbf{Range} & \textbf{DATA LSB}  & \textbf{Range}\\
        0 & Instrument & Part    & 0 to 63 & index   & 0 to 24 \\
        1 & Patch Set  & (none)  & 0       & index   & 0 to 24 \\
        2 & Scale      & (none)  & 0       & index   & 0 to 24 \\
        3 & State      & (none)  & 0       & index   & 0 to 24 \\
        4 & Vector     & Channel & 0 to 15 & index   & 0 to 24 \\
        5 & Midi Learn & (none)  & 0       & index   & 0 to 24 \\
      \end{tabular}
   \end{table}

   The absolute maximum size of these lists is 24, but if an index is given that
   is greater than the current list size an error will be reported.

   If an instrument part number is greater than 63 then it will be loaded to any
   part previously seen by the NRPN system, but if none was seen then the entire
   command will be ignored, on the basis that no action is better than wrong action.

   If a vector channel value is greater than 15 then \textsl{Yoshimi} will load this
   to the channel that the vector was originally saved from.


   \textsl{Yoshimi} can run with neither GUI nor CLI input access. Working purely as
   a hidden MIDI device.

   To enable a tidy close, there is a new short-form NRPN.
   Just send 68 to both MSB and LSB (CC 99 and CC 98). If you make the LSB 69, it
   will return an exit value of 16. This is the 'forced exit' and can be used by the
   system for other cleanups and a shut down.
\iffalse
\subsection{CC and NRPN Quick Reference}
\label{subsection:cc_nrpn_quick_ref}
   The table \tableref{table:CCandNRPNreference}, is a simple list of all the
   Continuous Controllers (CCs) and Non Registered Parameter Numbers (NRPNs)
   recognised by Yoshimi.
   For more CC details see \sectionref{subsubsec:concepts_midi_messages}, and for NRPN details \sectionref{sec:nrpns}.


\index{new!CC and NRPN Quick Reference}
      \begin{table}[H]
      \centering
      \caption{CC and NRPN Quick Reference}
      \label{table:CCandNRPNreference}
      \begin{tabular}{c c l}
    \textbf{CC} & & \textbf{Function} \\
       Hex (Dec) \\
       00 (0) & & Bank MSB\\
       01 (1) & & Modulation Wheel\\
       02 (2) & & Breath\\
       06 (6) & & NRPN Data MSB\\
       07 (7) & & Volume\\
       0A (10) & & Panning\\
       0B (11) & & Expression\\
       20 (32) & & Bank LSB\\
       26 (38) & & NRPN Data LSB \\
       40 (64) & & Sustain Pedal \\
       41 (65) & & Portamento \\
       44 (68) & & Legato Pedal \\
       47 (71) & & Filter Q \\
       4A (74) & & Filter Cutoff \\
       4B (75) & & Bandwidth \\
       4C (76) & & FM Amplitude \\
       4D (77) & & Resonance Centre Frequency\\
       4E (78) & & Resonance Bandwidth \\
       60 (96) & & NRPN Data Increment \\
       61 (97) & & NRPN Data Decrement \\
       62 (98) & & NRPN LSB \\
       63 (99) & & NRPN MSB \\
       78 (120) & & All Sound Off \\
       79 (121) & & Reset All Controllers \\
       7B (123) & & All Notes Off\\
       \\
    \textbf{NRPN MSB} & \textbf{NRPN LSB} & \textbf{Function} \\
        Hex (Dec) & Hex (Dec)\\
        04 (4) & 00-04 (0-4)& System Effect number \\
        08 (8) & 00-08 (0-8)& Insertion Effect number \\
        40 (64) & 00 (0) & Direct part control \\
        40 (64) & 01 (1) & Vector Control \\
        40 (64) & 02 (2) & System Settings \\
        60 (96) & 00-05 (0-5) & Load Numbered File from List \\
        \\
    \textbf{ShortForm} &  LSB is value & No data bytes needed \\
        41 (65) & 00-05 (0-5) & Set Solo Switch Type \\
        42 (66) & 00-77 (0-119) & Set Solo Switch Incoming Controller \\
        44 (68) & 44 (68) & Perform Normal Shutdown, exit value 0 \\
        44 (68) & 45 (69) & Perform shutdown without messages, exit value 16 \\
      \end{tabular}
   \end{table}
\fi


%-------------------------------------------------------------------------------
% vim: ts=3 sw=3 et ft=tex
%-------------------------------------------------------------------------------
