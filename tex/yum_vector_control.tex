%-------------------------------------------------------------------------------
% yum_vector_control
%-------------------------------------------------------------------------------
%
% \file        yum_vector_control.tex
% \library     Documents
% \author      Chris Ahlstrom
% \date        2015-10-24
% \update      2015-10-24
% \version     $Revision$
% \license     $XPC_GPL_LICENSE$
%
%     Provides the concepts NRPNs and vector control.  And effects.
%
%-------------------------------------------------------------------------------

\section{Vector Control}
\label{sec:vector}

   This section comes from the source-code documentation file
   \texttt{doc/Vector\_Control.txt}.

\subsection{Vector / Basics}
\label{subsection:vector_basics}

   Vector control is a way to control more than one part with the
   controllers.  It is a little bit reminiscent of the "vector" control knob
   on the Yamaha PSS-790 consumer MIDI synthesizer.  Vector control is only
   possible if one has 32 or 64 parts active.

   Vector control has been extended so that there are four independent
   'features' that each axis can control. One is fixed as \textsl{Volume} (if
   enabled) but the other three can be any valid CC, and can also be
   reversed. The vector 'sweep' CCs are split out very early in the MIDI
   chain, and the new CCs created are fed back in before any other
   processing. The result of this is that once we eventually get MIDI-learn
   implemented, the control possibilities will expand dramatically.
   \textsl{(Will notes: "sorry about the extreme delay :(")}

   In vector mode parts will still play together but the vector controls can
   change their volume, pan, filter cutoff in pairs, controlled by
   user-defined CCs set up with NRPNs.

   One must set the X axis CC before the Y axis, but if one doesn't set the
   Y axis at all, one can run just a single axis.
   If one has only 32 parts active, Y settings are ignored.

   For example:
   parts 1 and 17 can be set as x1 \& x2 (volume only) while parts 33 and 49
   can be y1 \& y2 (pan only).

   Independently of this Parts 2 \& 18 could use filter and pan from another
   CC.

\subsection{Vector / Vector Control}
\label{subsection:vector_control}

   Setting up vector control is currently done as follows.

   In the required channel send:

   \begin{itemize}
      \item NRPN MSB (99) set to 64
      \item NRPN LSB (98) set to 1 [8192]
      \item Data MSB (6) set mode:
      \begin{itemize}
         \item 0 = X CC
         \item 1 = Y CC
         \item 2 = X features
         \item 3 = Y features
         \item 4 = x1 instrument (optional)
         \item 5 = x2 instrument (optional)
         \item 6 = y1 instrument (optional)
         \item 7 = y2 instrument (optional)
      \end{itemize}
   \end{itemize}

   Setting CC for X enables vector control; any value outside the above list
   disables it.

   Data LSB (38) value to set features:

   \begin{itemize}
       \item 1 = Volume (fixed)
       \item 2 = Pan (the default)
       \item 4 = Filter Cutoff (Brightness, it is the default)
       \item 8 = Mod Wheel (the default)
       \item 0x12 = 18 = Reversed Pan
       \item 0x24 = 36 = Reversed Filter Cutoff
       \item 0x48 = 72 = Reversed Mod Wheel
   \end{itemize}

   The feature numbers are chosen so they can be combined. So, 5 would be
   Volume + Brightness and 19 would be Volume + Reversed Pan.

   Setting the sweep CC for the X axis enables vector control. It also sets,
   but doesn't enable the default X axis features.  Setting the sweep CC for
   the Y axis sets, but doesn't enable the default Y axis features.  If you
   don't enable any features, not a lot will happen.

   Optional settings.  The first part, the number, is the MSB value.
   The second part is the LSB, the parameter value to set.  Note that the
   instrument IDs are for instruments in the current bank.

   \begin{itemize}
      \item 4 = x1 instrument ID
      \item 5 = x2 instrument ID
      \item 6 = y1 instrument ID
      \item 7 = y2 instrument ID
      \item 8 = set CC for X feature 2
      \item 9 = set CC for X feature 4
      \item 10 = set CC for X feature 8
      \item 11 = set CC for Y feature 2
      \item 12 = set CC for Y feature 4
      \item 13 = set CC for Y feature 8
   \end{itemize}
              
Any data MSB value outside the above list disables vector control.

Sweep CCs and feature CCs are sanity checked.
    
   An Example: From channel 1, send the following CCs

   \begin{verbatim}
      CC      Value
      99       64
      98        1
       6        0
      38       14
      98        1 *
       6        1
      38       15
      98        1 *
       6        2
      38        1
      98        1 *
       6        3
      38        2
   \end{verbatim}

   This sequence will set up CC 14 as the X axis incoming controller,
   and CC 15 as the Y axis incoming controller, with X set to volume control
   and Y set to pan control.

   One can either go on with the NRPNs to set the instruments (this will load
   and enable instruments from the current bank), or enable and load
   them by hand.  For channel 1 this would be part 1 and 17 for X and part 33
   and 49 for Y.

   The (*) CCs ensure that the data bytes are reset each time. This is not
   really necessary for the earlier commands, but should be done if one sets
   the instruments with NRPNs as well, otherwise one will try to set them
   twice.

%-------------------------------------------------------------------------------
% vim: ts=3 sw=3 et ft=tex
%-------------------------------------------------------------------------------
